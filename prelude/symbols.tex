\newenvironment{centered}{\centering}{\par}

\definecolor{highlight}{RGB}{230,255,230}
\definecolor{darkgreen}{RGB}{76,170,76}
\definecolor{lavender}{RGB}{162,85,162}

\newcommand{\mathcolorbox}[2]{\colorbox{#1}{$\displaystyle #2$}}
\newcommand{\hlfancy}[2]{\sethlcolor{#1}\hl{#2}}
\makeatletter
\renewcommand*{\@textcolor}[3]{%
  \protect\leavevmode
  \begingroup
    \color#1{#2}#3%
  \endgroup
}
\makeatother

\newcommand{\subsubsubsection}[1]{\paragraph{\textbf{\textit{#1}}}}

\theoremstyle{plain}
\newtheorem*{theorem}{Theorem}
\newtheorem*{lemma}{Lemma}

\newtheorem{theoremn}{Theorem}[section]
\newtheorem{lemman}[theoremn]{Lemma}

\theoremstyle{definition}
\newtheorem*{definition}{Definition}
\newtheorem*{base}{Base}
\newtheorem*{induction}{Induction}
\newtheorem*{fact}{Fact}

\newcommand{\rulen}[1]{~~\text{(#1)}}
\newcommand{\judgment}[3]{\inferrule{#1}{#2}\rulen{\textsc{#3}}}
\newcommand{\judgbox}[1]{\noindent \fbox{$#1$}}

% H-types
\newcommand{\HTyp}{\texttt{HTyp}}
\newcommand{\HTypCtx}{\dot{\Gamma}}
\newcommand{\HTypVar}{\dot{\tau}}
\newcommand{\HTypVarP}{\HTypVar'}
\newcommand{\HTypNum}{\texttt{num}}
\newcommand{\HTypStr}{\texttt{str}}
\newcommand{\HTypArrow}[2]{(#1 \to #2)}

% H-expressions
\newcommand{\HExp}{\texttt{HExp}}
\newcommand{\HExpVar}{\dot{e}}
\newcommand{\HExpVarP}{\dot{e}'}
\newcommand{\HExpStrLit}{\texttt{"}\underline{s}\texttt{"}}
\newcommand{\HExpStrSub}[3]{#1\texttt{[}#2\texttt{:}#3\texttt{]}}

% Z-types
\newcommand{\ZTypVar}{\hat{\tau}}
\newcommand{\ZTypVarP}{\ZTypVar'}

% Z-expressions
\newcommand{\ZExp}{\texttt{ZExp}}
\newcommand{\ZExpVar}{\hat{e}}
\newcommand{\ZExpVarP}{\ZExpVar'}
\newcommand{\ZExpCursor}[1]{\mathcolorbox{highlight}{\textcolor{darkgreen}{\bm{\triangleright}}#1\textcolor{darkgreen}{\bm{\triangleleft}}}}
\newcommand{\ZExpCursorEHole}{\ZExpCursor{\ZExpEHole}}
\newcommand{\ZExpNEHole}[1]{\textcolor{lavender}{\bm{\llparenthesis}}#1\textcolor{lavender}{\bm{\rrparenthesis}}}
\newcommand{\ZExpEHole}{\ZExpNEHole{}}
\newcommand{\ZExpStrSub}[3]{\ZExpCursor{\HExpStrSub{#1}{#2}{#3}}}
\newcommand{\ZExpStrSubI}[3]{\HExpStrSub{\ZExpCursor{#1}}{#2}{#3}}
\newcommand{\ZExpStrSubII}[3]{\HExpStrSub{#1}{\ZExpCursor{#2}}{#3}}
\newcommand{\ZExpStrSubIII}[3]{\HExpStrSub{#1}{#2}{\ZExpCursor{#3}}}

% IH-expressions
\newcommand{\IHExp}{\texttt{IHExp}}
\newcommand{\IHExpVar}{d}
\newcommand{\IHExpVarP}{\IHExpVar'}
\newcommand{\IHExpCtx}[2]{#1 \dashv #2}
\newcommand{\IHExpNumLit}{\underline{n}}
\newcommand{\IHExpNumLitS}[1]{\underline{n_#1}}
\newcommand{\IHExpNumLitV}{n}
\newcommand{\IHExpNumLitVS}[1]{n_#1}
\newcommand{\IHExpStrLit}{\texttt{"}\underline{s}\texttt{"}}
\newcommand{\IHExpStrLitS}[1]{\texttt{"}\underline{s_#1}\texttt{"}}
\newcommand{\IHExpStrLitV}[1]{\texttt{"}s\texttt{"}}
\newcommand{\IHExpStrLitVS}[1]{\texttt{"}s_#1\texttt{"}}
\newcommand{\IHExpStrSub}[3]{#1\texttt{[}#2\texttt{:}#3\texttt{]}}

% IH hole context
\newcommand{\IHExpHoleCtx}{\Delta}
\newcommand{\IHExpHoleCtxE}{\cdot}

% Action expressions
\newcommand{\AcSym}{\alpha}
\newcommand{\AcExpArrow}[2]{#1 \stackrel{\AcSym}{\longrightarrow} #2}
\newcommand{\AcExpMovChild}[3]{#2 \xrightarrow{\texttt{move child #1}} #3}
\newcommand{\AcExpMovParent}[2]{#1 \xrightarrow{\texttt{move parent}} #2}
\newcommand{\AcCon}[3]{#2 \xrightarrow{\texttt{construct #1}} #3}
\newcommand{\AcConStrLit}[2]{\AcCon{lit $\HExpStrLit$}{#1}{#2}}
\newcommand{\AcConStrSub}[2]{\AcCon{subscript}{#1}{#2}}

% Elaboration expressions
\newcommand{\ElabArrow}[2]{#1 \leadsto #2}

% Evaluation expressions
\newcommand{\EvalCtx}{\mathcal{E}}
\newcommand{\EvalCtxEx}[2]{#1 = \EvalCtx\{#2\}}
\newcommand{\EvalCtxExF}[3]{#1 = #2\{#3\}}
\newcommand{\EvalTran}[2]{#1 \longrightarrow #2}

% Type rule expressions
\newcommand{\contextExpr}[2]{\ensuremath{#1 \vdash #2}}
\newcommand{\isConsistent}[2]{\ensuremath{#1 \sim #2}}
\newcommand{\isNotConsistent}[2]{\ensuremath{#1 \nsim #2}}
\newcommand{\hasType}[3]{\ensuremath{#1 \vdash #2 : #3}}
\newcommand{\hasTypeC}[2]{\ensuremath{#1 : #2}}
\newcommand{\hasTypeD}[4]{\ensuremath{#1; #2 \vdash #3 : #4}}
\newcommand{\hasTypeDX}[3]{\ensuremath{\hasTypeD{\Delta}{#1}{#2}{#3}}}
\newcommand{\synType}[3]{\ensuremath{#1 \vdash #2 \Rightarrow #3}}
\newcommand{\synTypeC}[2]{\ensuremath{#1 \Rightarrow #2}}
\newcommand{\synTypeD}[4]{\ensuremath{#1; #2 \vdash #3 \Rightarrow #4}}
\newcommand{\synTypeDX}[3]{\ensuremath{\synTypeD{\Delta}{#1}{#2}{#3}}}
\newcommand{\anaType}[3]{\ensuremath{#1 \vdash #2 \Leftarrow #3}}
\newcommand{\anaTypeC}[2]{\ensuremath{#1 \Leftarrow #2}}
\newcommand{\anaTypeD}[4]{\ensuremath{#1; #2 \vdash #3 \Leftarrow #4}}
\newcommand{\anaTypeDX}[3]{\ensuremath{\anaTypeD{\Delta}{#1}{#2}{#3}}}

\newcommand{\isGround}[1]{#1~\mathsf{ground}}
\newcommand{\isFinal}[1]{#1~\mathsf{final}}
\newcommand{\isVal}[1]{#1~\mathsf{val}}
\newcommand{\isIndet}[1]{#1~\mathsf{indet}}

\newcommand{\optPremise}[1]{\textcolor{red}{\bm{[}}#1\textcolor{red}{\bm{]}}}
